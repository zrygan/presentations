\subsection{Particle Errors \emph{with some persuasion} is Regular}\label{sec:particle_err}

\begin{frame}{Enclitic Particles in Tagalog}
    \textbf{Enclitic Particles (EP)} add meaning to preceding words.

    \vspace{3mm}

    \textbf{Examples:}
    \begin{itemize}
        \item ``rin'' / ``din'' --- inclusion (``too'', ``also'')
        \item ``raw'' / ``daw'' --- reported speech (``supposedly'')
    \end{itemize}

    \vspace{5mm}

    \textbf{Morphophonemic Rule:}
    \begin{itemize}
        \item Use \textbf{r-form} if preceding word ends in vowel (patinig)
        \item Use \textbf{d-form} if preceding word ends in consonant (katinig)
    \end{itemize}
\end{frame}

\begin{frame}{Examples: Reported Speech}
    \textbf{Correct usage:}

    \vspace{3mm}

    \begin{enumerate}
        \item Marami \textbf{raw} --- ends in vowel \emph{i}
        \item Humingi \textbf{raw} --- ends in vowel \emph{i}
        \item Sinapak \textbf{daw} --- ends in consonant \emph{k}
        \item Natulog \textbf{daw} --- ends in consonant \emph{g}
    \end{enumerate}
\end{frame}

\begin{frame}{Can We Use Feature Tagging?}
    Reported speech EPs can follow multiple parts of speech:
    \begin{itemize}
        \item Nouns ($\mathbb{NOUN}$)
        \item Verbs ($\mathbb{VERB}$)
        \item Adjectives ($\mathbb{ADJ}$)
        \item Adverbs ($\mathbb{ADV}$)
    \end{itemize}

    \vspace{5mm}

    \textbf{Yes!} Similar to~\cref{sec:spelling_err}, we can construct a regular grammar with feature tagging.
\end{frame}

\begin{frame}{Regular Grammar with Feature Tagging}
    \footnotesize
    \begin{align*}
        S               & \to \mathbb{NOUN}_p\,\text{raw}
        \;\mid\; \mathbb{NOUN}_k\,\text{daw}                                 \\
                        & \quad\mid\; \mathbb{VERB}_p\,\text{raw}
        \;\mid\; \mathbb{VERB}_k\,\text{daw}                                 \\
                        & \quad\mid\; \mathbb{ADJ}_p\,\text{raw}
        \;\mid\; \mathbb{ADJ}_k\,\text{daw}                                  \\
                        & \quad\mid\; \mathbb{ADV}_p\,\text{raw}
        \;\mid\; \mathbb{ADV}_k\,\text{daw}                                  \\
        \mathbb{NOUN}_p & \to \text{Lola} \mid \ldots                        \\
        \mathbb{NOUN}_k & \to \text{Juan} \mid \text{Bert} \mid \ldots       \\
        \mathbb{VERB}_p & \to \text{Humingi} \mid \text{Umiiyak} \mid \ldots \\
        \mathbb{VERB}_k & \to \text{Kumain} \mid \text{Sumigaw} \mid \ldots  \\
        \mathbb{ADJ}_p  & \to \text{Masaya} \mid \text{Maganda} \mid \ldots  \\
        \mathbb{ADJ}_k  & \to \text{Matulin} \mid \text{Mabilis} \mid \ldots \\
        \mathbb{ADV}_p  & \to \text{Ngayon} \mid \text{Dati} \mid \ldots     \\
        \mathbb{ADV}_k  & \to \text{Noon} \mid \text{Hapon} \mid \ldots
    \end{align*}
\end{frame}

\begin{frame}{The Problem: Combinatorial Explosion}
    \textbf{Issue:} Feature tagging creates production explosion!

    \vspace{5mm}

    For $n$ POS tags, each needing vowel/consonant split:
    \[
        \text{Total productions} = 2n + 1
    \]

    \vspace{5mm}

    \textbf{Consequence:}
    \begin{itemize}
        \item Grammar becomes unwieldy
        \item Difficult to maintain
        \item Defeats the purpose of ``easy to comprehend''
    \end{itemize}
\end{frame}

\begin{frame}{So\dots Is It Regular?}
    \textbf{Theoretically:} Yes! Morphophonemic alternation \emph{is} regular.

    \vspace{5mm}

    \textbf{Pragmatically:} Not really!

    \vspace{5mm}

    \textbf{The rule-based philosophy:}
    \begin{itemize}
        \item Leverage human comprehensibility
        \item Keep grammars maintainable
        \item Avoid spatial/practical inefficiency
    \end{itemize}

    \vspace{3mm}

    A spatially inefficient rule-based approach is a (essentially)
    \textbf{useless}.
\end{frame}
